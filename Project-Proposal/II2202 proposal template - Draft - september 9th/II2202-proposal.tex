\documentclass[12pt,twoside,english]{article}
\usepackage[utf8]{inputenc}

%%%%%%%%%%%%%%%%%%%%%%%%%%%%%%%%%%%%%%%%%%%%%%%%%%%%%%%%%%%%%%%%%%%%%%%
%% template for II2202 proposal
%% original 2020.08.28
%% revised  
%%%%%%%%%%%%%%%%%%%%%%%%%%%%%%%%%%%%%%%%%%%%%%%%%%%%%%%%%%%%%%%%%%%%%%%
%


\usepackage[paper=a4paper,dvips,top=1.5cm,left=1.5cm,right=1.5cm,
    foot=1cm,bottom=1.5cm]{geometry}


\usepackage{todonotes}          %to provide inline and margin notes
%\usepackage[T1]{fontenc}
%%\usepackage{pslatex}
\renewcommand{\rmdefault}{ptm} 
\usepackage{mathptmx}
\usepackage[scaled=.90]{helvet}
\usepackage{courier}
%
\usepackage{bookmark}

\usepackage{fancyhdr}
\pagestyle{fancy}

%%----------------------------------------------------------------------------
%%   pcap2tex stuff
%%----------------------------------------------------------------------------
 %%\usepackage[dvipsnames*,svgnames]{xcolor} %% For extended colors
 %%\usepackage{tikz}  %% Already loaded
 %%\usetikzlibrary{arrows,decorations.pathmorphing,backgrounds,fit,positioning,calc,shapes}

%% \usepackage{pgfmath}	% --math engine
%%----------------------------------------------------------------------------
%% \usepackage[latin1]{inputenc}
\usepackage[utf8]{inputenc} % inputenc allows the user to input accented characters directly from the keyboard
\usepackage[english]{babel}
%% \usepackage{rotating}		 %% For text rotating
\usepackage{array}			 %% For table wrapping
\usepackage{graphicx}	                 %% Support for images
\usepackage{float}			 %% Suppor for more flexible floating box positioning
\usepackage{color}                       %% Support for colour 
\usepackage{mdwlist}
%% \usepackage{setspace}                 %% For fine-grained control over line spacing
%% \usepackage{listings}		 %% For source code listing
%% \usepackage{bytefield}                %% For packet drawings
\usepackage{tabularx}		         %% For simple table stretching
%%\usepackage{multirow}	                 %% Support for multirow colums in tables
\usepackage{dcolumn}	                 %% Support for decimal point alignment in tables
\usepackage{url}	                 %% Support for breaking URLs
\usepackage[perpage,para,symbol]{footmisc} %% use symbols to ``number'' footnotes and reset which symbol is used first on each page
\usepackage{siunitx}
\newcommand{\SIadj}[2]{\SI[number-unit-product={\text{-}}]{#1}{#2}}

%% \usepackage{pygmentize}           %% required to use minted -- see python-pygments - Pygments is a Syntax Highlighting Package written in Python
%% \usepackage{minted}		     %% For source code highlighting

%% \usepackage{hyperref}		
\usepackage[all]{hypcap}	 %% Prevents an issue related to hyperref and caption linking
%% setup hyperref to use the darkblue color on links
%% \hypersetup{colorlinks,breaklinks,
%%             linkcolor=darkblue,urlcolor=darkblue,
%%             anchorcolor=darkblue,citecolor=darkblue}

%% Some definitions of used colors
\definecolor{darkblue}{rgb}{0.0,0.0,0.3} %% define a color called darkblue
\definecolor{darkred}{rgb}{0.4,0.0,0.0}
\definecolor{red}{rgb}{0.7,0.0,0.0}
\definecolor{lightgrey}{rgb}{0.8,0.8,0.8} 
\definecolor{grey}{rgb}{0.6,0.6,0.6}
\definecolor{darkgrey}{rgb}{0.4,0.4,0.4}
%% Reduce hyphenation as much as possible
\hyphenpenalty=15000 
\tolerance=1000

%% useful redefinitions to use with tables
\newcommand{\rr}{\raggedright} %% raggedright command redefinition
\newcommand{\rl}{\raggedleft} %% raggedleft command redefinition
\newcommand{\tn}{\tabularnewline} %% tabularnewline command redefinition

%% definition of new command for bytefield package
\newcommand{\colorbitbox}[3]{%
	\rlap{\bitbox{#2}{\color{#1}\rule{\width}{\height}}}%
	\bitbox{#2}{#3}}

%% command to ease switching to red color text
\newcommand{\red}{\color{red}}
%%redefinition of paragraph command to insert a breakline after it
\makeatletter
\renewcommand\paragraph{\@startsection{paragraph}{4}{\z@}%
  {-3.25ex\@plus -1ex \@minus -.2ex}%
  {1.5ex \@plus .2ex}%
  {\normalfont\normalsize\bfseries}}
\makeatother

%%redefinition of subparagraph command to insert a breakline after it
\makeatletter
\renewcommand\subparagraph{\@startsection{subparagraph}{5}{\z@}%
  {-3.25ex\@plus -1ex \@minus -.2ex}%
  {1.5ex \@plus .2ex}%
  {\normalfont\normalsize\bfseries}}
\makeatother

\setcounter{tocdepth}{3}	%% 3 depth levels in TOC
\setcounter{secnumdepth}{5}
%% Acronyms
\usepackage[acronym, nopostdot]{glossaries}
\glsdisablehyper
\makeglossaries

\renewcommand{\headrulewidth}{0pt}
%%%%%%%%%%%%%%%%%%%%%%%%%%%%%%%%%%%%%%%%%%%%%%%%%%%%%%%%%%%%%%%%%%%%
%% End of preamble
%%%%%%%%%%%%%%%%%%%%%%%%%%%%%%%%%%%%%%%%%%%%%%%%%%%%%%%%%%%%%%%%%%%%
\usepackage{verbatim}
\newcommand{\detailtexcount}[1]{%
  \immediate\write18{texcount -merge -sum -q #1.tex output.bbl > #1.wcdetail }%
  \verbatiminput{#1.wcdetail}%
}
\newcommand{\quickwordcount}[1]{%
  \immediate\write18{texcount -1 -sum -merge -q #1.tex output.bbl > #1-words.sum }%
  \input{#1-words.sum} words%
}
\newacronym{HEP}{HEP}{High Energy Physics}



\title{A Review on Text Data Format for Data Science}
%Text Data Formats in Data Science: Performance Evaluation}
\author{
        \textsc{Steven Shidi Zhou}
            \qquad
        \textsc{Emil Ståhl}
        \mbox{}\\
        \normalsize
            \texttt{shidi}
        \textbar{}
            \texttt{emilstah}
        \normalsize
            \texttt{@kth.se}
}
\date{\today}

\lhead{II2202, Fall 2022, Period 1}
%% or \lhead{II2202, Fall 2020, Period 1}
\chead{Project Proposal}
\rhead{\date{\today}}

\makeatletter
\let\ps@plain\ps@fancy 
\makeatother

\setlength{\headheight}{15pt}
\begin{document}
%TC:ignore
\maketitle
%TC:endignore

% \begin{abstract}
% \label{sec:abstract}

% Your abstract here.

% \end{abstract}
%%\clearpage

\selectlanguage{english}
%TC:ignore
%TC:endignore

\section{Allocation of responsibilities}
\label{sect:alloc_responsibilities}
Steven Shidi Zhou is responsible for writing the benchmark programs and systematically testing the data set formats in question, writing the Introduction, Background, and Results Analysis Section in the report, and presenting the project plan. Additionally, Emil Ståhl is responsible for collecting data using the benchmark programs produced by Steven Shidi Zhou, conducting a qualitative assessment, writing the Results and Conclusion sections in the report, and presenting the ethical and sustainability aspects of the project. Both authors are responsible for presenting the final project and the final version of the report.


\section{Organization}
\label{sect:organization}
This project is organized as a two-person project and build on the basis of some previous works, such as \cite{plase_comparison_2017} and \cite{blomer_quantitative_2018}. Qualitative and quantitative assessments will run simultaneously.


\section{Background}
\label{sect:background}
The modern society produces a vast amount of data every day across many different types of industries and pipelines. Until processed, this data is merely bits placed on a storage medium. The field of data science is focusing on extracting useful knowledge from such data with the goal of gaining a deeper understanding of a given field. Data can be stored in various different formats such as csv, xlsx, Parquet, and Avro. Processing this data requires a lot of operations to be applied such as read, write, and other mathematical operations. Given the value that data science brings puts heavy requirements on the performance and stability of the data formats used in the processing and storage pipelines. Related work has been made by Plase et al. which in their work "A comparison of HDFS compact data formats: Avro versus Parquet" have benchmarked the amount of disk space these formats consumes as well as their performance in \gls{HEP} analysis which uses lots of numerical data. However, the work does not benchmark read/write and simple operations, nor does it benchmark the performance on text data. Understanding the performance advantages of different data formats in terms of read/write speed on text data is therefore of paramount importance in the data science community\cite{plase_comparison_2017}.


\begin{comment}
This project builds on the basis of previous work's comparison on different data formats, such as comparing the compact data formats like Avro and Parquet in HDFS, or data formats for a specific field such as \gls{HEP} analysis. \cite{plase_comparison_2017, blomer_quantitative_2018}.

%Adding some more content
We will use the quantitative method introduced from before those previous work, such as the file stability test using bit flip, but with the time constraint, we will only be performing the benchmark against 3 or 4 common file formats (csv, parquet, avro, and xlsx if time permits) used in Data Science.
\end{comment}

\section{Problem statement}
\label{sect:problem_statement}
Analysing the performance and stability of different data formats is key to tackle the scalability and performance challenges in which data science pipelines operates. Today, there exists over 44 zettabytes of data with around 2.5 quintillion bytes worth of data generated each day. (source https://explodingtopics.com/blog/big-data-stats). Data science pipelines must be highly optimized performance-wise in order to be able to process data at sufficient speeds and to ensure stability and job efficiency. Since the different data formats are implemented using different technologies, they may differ in how they handle different sizes of data which affects the performance when processing and storing the data. There are currently no conventions for when to use a particular data format for a given data size to optimize for performance and file stability. Since csv, xlsx, Parquet, and Avro are all widely used in data science it is important to have an understanding of their behavior.  

%Adding some more content
\begin{comment}
This project will benchmark 3 (or 4 if time permits) different text data formats commonly used in the field of Data Science, using open licensed data of different sizes (smaller than 1 MB, between 1 MB and 100 MB, bigger than 500 MB). We will then evaluate their performance according to the 3 different benchmarking metrics (file stability, read/write speed, and finally simple operation such sum items). The benchmark will be performed using Python; for this project, we will focus on just one language instead of all the other possible languages; the reason for this is because Python is one of the most used programming languages in Data Science.
\end{comment}

\section{Purpose}
\label{sect:purpose}
The purpose of this work is to analyze the performance of the csv, xlsx, Parquet, and Avro data formats by quantifying their read/write speed, file stability, and other operations such as summation. By analysing the performance of these data formats in different pipeline conditions, such as file size, one can use this work for tailoring data science pipelines in order to optimize for performance. The results are potentially of interest for general data scientists, developers of big data frameworks, and the open source community.

\begin{comment}
The purpose of this work is to present various metrics and criteria to help actors make the correct choice regarding data formats according to their needs. By analyzing the performance of data formats for data science under different conditions, one can use this work to tailor data science pipelines in order to minimize disk and network resources to optimize performance. The results are potentially of interest to research organizations, commercial actors, and the open-source community related to the field of data science.
\end{comment}

\section{Research question and contributions to the state of the art}
Considering the background and the problem discussion, the paper answers the following research question:\\

% can be changed, just brainstorming
\textbf{RQ1} How does the csv, xlsx, Parquet, and Avro data formats differ in read/write performance on as well as resilience against file errors when storing text data?\\

The main contributions of our work are:

\begin{itemize}
    \item 1
    \item 2
\end{itemize}


\section{Hypothesis}
\label{sect:hypothesis}
There is an advantageous data format for a given file size, a row-based Avro format should provide advantages performance with regard to simple operations such as read/write speeds when compared to csv, xlsx, and Parquet. For larger data sets, the Apache Parquet format should provide better performance.


\section{Goal}
\label{sect:goals}
The goal of this work is to run a set of experiments where data of various sizes stored in the csv, xlsx, Parquet, and Avro formats is processed and exposed to random bit flips. The experiments are going to measure metrics such as read/write speed and file stability with the goal of producing an analytical model showing when a particular data format may be advantageous. This model can be used to invalidate the hypothesis. The results are then going to be described, analyzed, and discussed in order to determine if there are any bottlenecks regarding the different data formats ability to ensure performance and resilience against file errors. 


\section{Tasks}
\label{sect:tasks}
A collection of different data with various sizes will be converted into different data formats such as csv, xlsx, Parquet, and Avro. Our test will be measuring the time it takes to read/write such data, as well as some basic transform operations, with each of the data formats. To measure how well the different data formats handle error, a file stability experiment will be conducted by randomly flipping file bits to simulate a corrupt data file. \cite{blomer_quantitative_2018}. Finally, qualitative methods will be prepared to gain more insight into how popular or user friendly each format is.

\section{Method}
\label{sect:method}
The project will use the empirical method \cite{blomer_quantitative_2018} for the data format stability experiment, as well as the general performance test. We, the authors, will also use an analytical method (e.g., interview or survey) to try to get a full image; however, the analytic part of this study will be shortened due to limited time period and resource of the assignment. 

\section{Milestone chart (time schedule)}
\label{sect:milestones}
The project will start on 7 September and end at 16:59 on 28 October. There will be the following milestones and deliverables:

\begin{description}
\item{\textbf{8 September}} Presentation of the proposed research: Ethics \& Sustainability.

\item{\textbf{14 September}} Research plan: First draft of the research plan, presentation, and peer review.

\item{\textbf{21 September}} Completion of benchmark programs.

\item{\textbf{28 September}} Quantitative analysis done.

\item{\textbf{5 October}} Result collection and analysis done (Qualitative interview).

\item{\textbf{10 October}} Final report: First draft and presentation with peer review of the draft report and presentation.

\item{\textbf{14 October}} Written opposition: before the final seminar - with peer review.

\item{\textbf{28 October}} Final report submission (the report will have been written in parallel with each of the above steps).


\end{description}

\clearpage
\bibliography{Data-format}
%%\bibliographystyle{IEEEtran}
\bibliographystyle{myIEEEtran}

\section{Acronyms}
%TC:ignore
\renewcommand{\glossarysection}[2][]{} %% skip the title
\printglossary[type=\acronymtype,nonumberlist]
%TC:endignore
\clearpage
%%TC:ignore
%\detailtexcount{II2202-proposal}
%%TC:endignore
\end{document}

